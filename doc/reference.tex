\documentclass[a4paper,NoNotes,GeneralMath]{stdmdoc}

\begin{document}
	\title{Bèzier Approximating an Image}
	\author{Dario Balboni}	
	\autodate

	\section*{Definitions}
	\subsection*{Fat Bèzier Curve}
	We define a Bèzier Curve of degree $n$ to be the image of the function $\phi: [0, 1] \rar \bbR^2$ defined by
	$$ \phi(t) = (\sum_{i=0}^n a_i \binom{n}{i} t^i (1-t)^{n-i},  \sum_{i=0}^n b_i \binom{n}{i} t^i (1-t)^{n-i}) $$
	where the $a_i$ and $b_i$ are parameters of the curve. A Fat Bèzier Curve is the set of points in $\bbR^2$ which
	distance from $\Img \phi$ is less then a defined quantity $l$.

	\section*{Descriptions of Structures}
	\subsection*{Fat Bèzier Curve}
	We represent a Fat Bèzier Curve in a row matrix with the following structure:
	$$ \left( \begin{array}{ccccccc} l & a_0 & \ldots & a_n & b_0 & \ldots & b_n \\ \end{array} \right) $$
	with notation as above.

	\section*{Description of Methods}
	\subsection*{De Casteljau algorithm and curve splitting}
	Given a Bèzier Curve of degree $n$, it is possible to evaluate the curve at the time $t_0$ and split it in two
	curves with an algorithm taking only $n$ steps. We set the recurrence relation:
	$$ \beta_i^{(0)} := \beta_i \qquad i = 0, \ldots, n $$
	$$ \beta_i^{(j)} := \beta_i^{(j-1)} (1 - t_0) + \beta_{i+1}^{(j-1)} t_0 \qquad i = 0, \ldots, n-j, \quad j = 1 , \ldots, n $$
	where the $\beta_i$ are multi-coordinate points. The evaluation of the Bèzier curve at time $t_0$ is $B(t_0) = \beta_0^{(n)}$
	and the curve can be split into two curves with control points respectively: $\beta_0^{(0)}, \beta_0^{(1)}, \ldots, \beta_0^{(n)}$
	and $\beta_0^{(n)}, \beta_1^{(n-1)}, \ldots, \beta_n^{(0)}$. \\

	\section*{Description of Procedures}
	\subsection*{DistanceFromPoint}
	We use the De Casteljau algorithm for splitting a Bèzier curve in halfes recursively, and taking the minimum of the
	distances on the two halves. When the piece of curve is almost a line, we calculate the minimum point distance and return. \\
	We then base on paper {\it bez.pdf} to measure the flatness of a curve and to do the appropriate math for a line segment.

\end{document}
